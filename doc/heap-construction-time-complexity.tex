\documentclass[11pt, oneside]{article}   	% use "amsart" instead of "article" for AMSLaTeX format
%\usepackage{geometry}                		% See geometry.pdf to learn the layout options. There are lots.
%\geometry{letterpaper}                   		% ... or a4paper or a5paper or ... 
%\geometry{landscape}                		% Activate for rotated page geometry
%\usepackage[parfill]{parskip}    		% Activate to begin paragraphs with an empty line rather than an indent

\usepackage{geometry}
 \geometry{
 a4paper,
 total={170mm,257mm},
 left=20mm,
 top=20mm,
 bottom=20mm,
 }

\usepackage{graphicx}				% Use pdf, png, jpg, or eps§ with pdflatex; use eps in DVI mode
								% TeX will automatically convert eps --> pdf in pdflatex		
\usepackage{amssymb}
\usepackage{amsmath}
\usepackage{fancyhdr}
\usepackage[utf8]{inputenc}
\usepackage[english]{babel}
\usepackage{enumerate}
\usepackage{arcs}
\usepackage{subfiles}

%SetFonts

%SetFonts

\usepackage[inline]{asymptote}


\pagestyle{fancy}
\fancyhf{}
\rhead{David Zhong}
\lhead{\leftmark}
\lfoot{Copyright \copyright  2019 - 2022 by David Zhong. All rights reserved.}
\rfoot{Page \thepage}

\title{Heap Construction Algorithm and Time Complexity Analysis}
\author{David Zhong}
%\date{}							% Activate to display a given date or no date
\begin{document}
\maketitle

\section{HeapSort Implemented in Python 3}
\begin{verbatim}
class HeapSort:
    def __init__(self, array):
        self.array = array
        self.length = len(array)

    def swap(self, i, j):
        tmp = self.array[i]
        self.array[i] = self.array[j]
        self.array[j] = tmp

    def adjust_heap(self, i, n):
        while 2*i + 1 < n:
            child = 2*i + 1
            
            if child + 1 < n and self.array[child + 1] > self.array[child]:
                child = child + 1

            if self.array[i] >= self.array[child]:
                return
            
            self.swap(i, child)
            i = child
        
    def construct_heap(self):
        i = (self.length - 1) // 2
        while i >= 0:
            self.adjust_heap(i, self.length)
            i = i - 1

    def sort(self):
        self.construct_heap()
        for i in range(0, self.length):
            self.swap(0, self.length-i-1)
            self.adjust_heap(0, self.length-i-1)

\end{verbatim}

\newpage
\section{Time Complexity Analysis for Heap Construction}
Essentially, we view all nodes in a binary tree structure: at level 0, there is only one node, e.g., the 1st node in the array;  there are at most $2^k$ nodes at level $k$.

Suppose we have $n = 2^{k+1}-1$ elements in the array, and these $n$ nodes just form a binary tree with $(k+1)$ levels: level 0, level 1, ..., level $k$. Starting from the last element in level $(k-1)$, the construction algorithm takes it as a parent node and does heap adjustment. There are $2^{k-1}$ elements in the level $(k-1)$. For every node in this level, the heap adjustment is just for the subtree formed by the parent and its two children, and the height of such a subtree is 1.

The complexity comes from the fact that, starting from level $(k-2)$ to level 0, the heap adjustment for any parent node $y$ might have to adjust the whole subtree rooted at $y$, and the time complexity of this adjustment is $O(h)$, where $h$ is the height of the subtree.

The time complexity for heap construction is therefore:
\[T = 2^{k-1} \cdot 1 + 2^{k-2}\cdot 2 + 2^{k-3}\cdot 3 + \cdots + 1\cdot k.\]

To calculate $T$, we have
\[2T = 2^k \cdot 1 + 2^{k-1}\cdot 2 + 2^{k-2}\cdot 3 + \cdots + 2\cdot k,\]
and then,
\[T = 2T - T = 2^k + 2^{k-1} + \cdots + 2^1 - k = 2^{k+1} - k - 2 =  n - k -1 = O(n).\]


\end{document}  